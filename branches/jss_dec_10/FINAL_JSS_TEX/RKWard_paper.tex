\documentclass[article,shortnames]{jss}
%%%%%%%%%%%%%%%%%%%%%%%%%%%%%%
%% declarations for jss.cls %%%%%%%%%%%%%%%%%%%%%%%%%%%%%%%%%%%%%%%%%%
%%%%%%%%%%%%%%%%%%%%%%%%%%%%%%

%% almost as usual
\author{Stefan R\"odiger\\L.U.A.S. AND Charit\'e
	\And Prasenjit Kapat\\Plus Affiliation
	\And Meik Michalke\\Plus Affiliation
	\And Thomas Friedrichsmeier\\Ruhr-University Bochum}
\title{RKWard - A Comprehensive Graphical User Interface and Integrated Development Environment for Statistical Analysis With {R}}

%% for pretty printing and a nice hypersummary also set:
\Plainauthor{Stefan R\"odiger, Prasenjit Kapat, Meik Michalke, Thomas Friedrichsmeier} %% comma-separated
\Plaintitle{RKWard - A Comprehensive Graphical User Interface and Integrated Development Environment for Statistical Analysis With R} %% without formatting
\Shorttitle{RKWard a GUI to {R}} %% a short title (if necessary)

%% an abstract and keywords
\Abstract{
  R is a free, open-source implementation of the S statistical computing
language and programming environment. The current status of R is a
command line driven interface with no advanced standard Graphical User
Interface (GUI) but it includes tools for building such. Over the past
years proprietary and non-proprietary GUI solutions, based on internal
or external tool kits, with different scopes and technological concepts
have emerged. In this paper we discuss RKWard which aims to be both a
comprehensive cross-platform GUI and Integrated Development Environment
(IDE) for R. RKWard is based on the KDE software libraries. Statistical
procedures and plots are implemented using an extendable plugin
architecture based on ECMA script (JavaScript), R, and XML. RKWard
provides an excellent tool to manage different types of data objects;
even allowing for seamless editing of certain types. The objective of
RKWard is to provide a portable and extensible R interface for both
basic and advanced statistical and graphical analysis while not
compromising on flexibility and modularity of the R programming
environment itself.
}
\Keywords{GUI, IDE, R, plugin, cross-platform}
\Plainkeywords{keywords, comma-separated, not capitalized, Java} %% without formatting
%% at least one keyword must be supplied

%% publication information
%% NOTE: Typically, this can be left commented and will be filled out by the technical editor
%% \Volume{13}
%% \Issue{9}
%% \Month{September}
%% \Year{2004}
%% \Submitdate{2004-09-29}
%% \Acceptdate{2004-09-29}

%% The address of (at least) one author should be given
%% in the following format:
\Address{
  Stefan R\"odiger\\
  Lausitz University of Applied Sciences (L.U.A.S)\\
  Department of Bio-, Chemistry and Process Engineering\\
  AND\\
  Center for Cardiovascular Research (CCR)\\
  Charit\'e, Germany\\
  E-mail: \email{stefan_roediger@gmx.de}
}

\Address{
  Prasenjit Kapat\\
  Affiliation\\
  Department\\
  E-mail: \email{noname@here.org}
}
\Address{
  Meik Michalke\\
  Affiliation\\
  Department\\
}

\Address{
  Thomas Friedrichsmeier\\
  Affiliation\\
  Department\\
  E-mail: \email{noname@here.org}
}
%% It is also possible to add a telephone and fax number
%% before the e-mail in the following format:
%% Telephone: +43/1/31336-5053
%% Fax: +43/1/31336-734

%% for those who use Sweave please include the following line (with % symbols):
%% need no \usepackage{Sweave.sty}

%% end of declarations %%%%%%%%%%%%%%%%%%%%%%%%%%%%%%%%%%%%%%%%%%%%%%%


\begin{document}

%% include your article here, just as usual
%% Note that you should use the \pkg{}, \proglang{} and \code{} commands.

\section[Abbreviations]{Abbreviations}
CRAN, The Comprehensive R Archive Network; CSS, Cascading Style Sheet;
CSV, Comma separated values; GUI, Graphical User Interface; HTML,
Hypertext Markup Language; IDE, Integrated Development Environment; JS,
JavaScript; KDE SC, KDE Software Compilation; ODF, OASIS Open Document
Format; PHP, PHP: Hypertext Preprocessor; PNG, Portable Network
Graphic; Rcmdr, R Commander; TDI, Tab document interface; XML, Extended
Markup Language

\section[Background and motivation]{Background and motivation}
In mid 1993 Ihaka and Gentleman published initial efforts on the computing
language and programming environment R on the s-news mailing list. Ambitions for
this project aimed to develop an S-like language but without inheriting memory
and performance issues. The source code of R was finally released in 1995 and
development has since evolved under the umbrella of the R Development Core Team
since mid 1997 (\cite {RDCT2001}, \cite{RDCT2010}; \cite{Ihaka_Gentlemen_1993}).
R does not include an advanced cross-platform GUI as known from other
statistical software packages. However, R includes tools for building GUIs
mainly based on Tlc/Tk (\cite{Dalgaard2001}, \cite{Dalgaard2002}). Since then a
plethora of R GUIs have emerged (see \url{http://www.sciviews.org/_rgui/} for a
comprehensive list). In 2005 John Fox released version 1.0 of R Commander which
can be considered a milestone in R GUI development; this was the first GUI
implementation which was able to deliver the experience of statistical tests,
plots and data manipulation easily accessible for R novices as well as advanced
users. However, John Fox stated that R Commander's target was to provide
functionality for basic-statistical courses though functionality increased over
time beyond this (\cite{Fox2005}). In November 2002 Thomas Friedrichsmeier
started the RKWard open-source software project with the goal to create an
implementation of an R GUI based on KDE and Qt technologies.

The scope of RKWard is deliberately broad, targeting both R novices and experts.
Regarding the first group, the aim is to allow any person with knowledge on
statistical procedures to start using RKWard for their everyday work,
immediately, without having to learn anything about the R programming language,
first. At the same time RKWard tries to support users who want to learn and
exploit the full flexibility of the R language for automating or customizing
analyses. At the other end of the learning curve, RKWard provides advanced IDE
features to R experts to assist in the development of R scripts. Yet, the idea
is that R experts, too will benefit from the availability task-oriented GUI
dialogs from time to time, such as when exploring an unfamiliar type of analysis
or by allowing to implement routinely performed tasks as a GUI element. In
addition, many features, such as the integrated data editor, or the plot preview
feature will be useful to R novices and R experts alike in their everyday work
(see section \ref{Default Graphical User Interface Elements}).

While RKWard tries to support users wishing to learn R, it is specifically not
designed as a teaching tool (such as Rcmdr or TeachingDemos), but as a
productive tool. This means that dialogs for statistical procedures in RKWard do
not necessarily show a 1:1 correspondence to the underlying steps in R, but are
rather oriented at statistical tasks. At the same time, RKWard provides a high
level of transparency about the steps that are needed to perform any supported
task in R. RKWard tries to make it easy for the user to see complete codes for
all GUI actions. In doing so RKWard accepts relatively verbose generated code,
deliberately, and in particular it does not attempt to wrap complex sequences of
data manipulation or analysis into custom high-level R functions. We believe the
task of providing such high-level functions is logically independent of the
development of a GUI frontend, and should best be solved in dedicated R
packages, where needed. Rather, RKWard limits itself to generate R code from GUI
settings. This allows to make better use of the modular design of R, avoids
locking in users to a specific GUI solution, and allows them more options for
customizing generated code patterns. Further, RKWard does not impose artificial
limitations on how users can work with the application. For example allowing to
use only one data.frame at a time is or to use data from several frames is
intentionally supported.

Finally, RKWard is designed to allow users to create custom GUI dialogs easily
(see sections \ref{technical_plugins} and \ref{example_plugins}).

RKWard is licensed under the terms of the GNU General Public License Version 2
or later. This means the RKWard code itself is GPL v 2 or 3 but effectively
distributable only under GPL v 2 due to R code. Some documentation templates are
GFDL licensed. The current project status of the core application is tagged as
beta (K2009). However, from experience RKWard is reliably usable in productive
scenarios. The source code, selected binaries and documentation is hosted at
SourceForge (http://sourceforge.net/). Milestones of the RKWards development are
demonstrated in Figure \ref{development_overview}.

In this paper we will first give an overview over the main GUI elements and features
of RKWard. Next some technical aspects of the implementation will be dicussed. The paper
concludes with two examples: One user centered example of a simple RKWard session, and
an example, targetted at developers, for creating a simple plugin extension to RKWard.


\bibliography{sources}
\end{document}
