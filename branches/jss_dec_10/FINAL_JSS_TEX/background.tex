\section[Background and motivation]{Background and motivation}
In mid 1993 Ihaka and Gentleman published initial efforts on the computing
language and programming environment R on the s-news mailing list. Ambitions for
this project aimed to develop an S-like language but without inheriting memory
and performance issues. The source code of R was finally released in 1995 and
development has since evolved under the umbrella of the R Development Core Team
since mid 1997 \citep{RDCT2001, RDCT2010, Ihaka_Gentlemen_1993}.
R does not include an advanced cross-platform GUI as known from other
statistical software packages. However, R includes tools for building GUIs
mainly based on Tlc/Tk \citep{Dalgaard2001, Dalgaard2002}. Since then a
plethora of R GUIs have emerged (see \url{http://www.sciviews.org/_rgui/} for a
comprehensive list). In 2005 John Fox released version 1.0 of R Commander which
can be considered a milestone in R GUI development; this was the first GUI
implementation which was able to deliver the experience of statistical tests,
plots and data manipulation easily accessible for R novices as well as advanced
users. However, John Fox stated that R Commander's target was to provide
functionality for basic-statistical courses though functionality increased over
time beyond this \citep{Fox2005}. In November 2002 Thomas Friedrichsmeier
started the RKWard open-source software project with the goal to create an
implementation of an R GUI based on KDE and Qt technologies.

The scope of RKWard is deliberately broad, targeting both R novices and experts.
Regarding the first group, the aim is to allow any person with knowledge on
statistical procedures to start using RKWard for their everyday work,
immediately, without having to learn anything about the R programming language,
first. At the same time RKWard tries to support users who want to learn and
exploit the full flexibility of the R language for automating or customizing
analyses. At the other end of the learning curve, RKWard provides advanced IDE
features to R experts to assist in the development of R scripts. Yet, the idea
is that R experts, too will benefit from the availability task-oriented GUI
dialogs from time to time, such as when exploring an unfamiliar type of analysis
or by allowing to implement routinely performed tasks as a GUI element. In
addition, many features, such as the integrated data editor, or the plot preview
feature will be useful to R novices and R experts alike in their everyday work
(see section \ref{Default Graphical User Interface Elements}).

While RKWard tries to support users wishing to learn R, it is specifically not
designed as a teaching tool (such as \pkg{Rcmdr} or \pkg{TeachingDemos}), but as a
productive tool. This means that dialogs for statistical procedures in RKWard do
not necessarily show a 1:1 correspondence to the underlying steps in R, but are
rather oriented at statistical tasks. At the same time, RKWard provides a high
level of transparency about the steps that are needed to perform any supported
task in R. RKWard tries to make it easy for the user to see complete codes for
all GUI actions. In doing so RKWard accepts relatively verbose generated code,
deliberately, and in particular it does not attempt to wrap complex sequences of
data manipulation or analysis into custom high-level R functions. We believe the
task of providing such high-level functions is logically independent of the
development of a GUI frontend, and should best be solved in dedicated R
packages, where needed. Rather, RKWard limits itself to generate R code from GUI
settings. This allows to make better use of the modular design of R, avoids
locking in users to a specific GUI solution, and allows them more options for
customizing generated code patterns. Further, RKWard does not impose artificial
limitations on how users can work with the application. For example allowing to
use only one data.frame at a time is or to use data from several frames is
intentionally supported.

Finally, RKWard is designed to allow users to create custom GUI dialogs easily
(see sections \ref{technical_plugins} and \ref{example_plugins}).

RKWard is licensed under the terms of the GNU General Public License Version 2
or later. This means the RKWard code itself is GPL v 2 or 3 but effectively
distributable only under GPL v 2 due to R code. Some documentation templates are
GFDL licensed. The current project status of the core application is tagged as
beta (K2009). However, from experience RKWard is reliably usable in productive
scenarios. The source code, selected binaries and documentation is hosted at
SourceForge (http://sourceforge.net/). Milestones of the RKWards development are
demonstrated in Figure \ref{development_overview}.

In this paper we will first give an overview over the main GUI elements and features
of RKWard. Next some technical aspects of the implementation will be dicussed. The paper
concludes with two examples: One user centered example of a simple RKWard session, and
an example, targetted at developers, for creating a simple plugin extension to RKWard.
