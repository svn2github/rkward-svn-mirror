\section{Installing and starting RKWard}
\label{sec:installing_starting_RKWard}
RKWard can be downloaded free of charge in source and binary form from
\url{http://sourceforge.net/}. On the GNU/Linux
platform, binary packages are available for many major distributions,
including Debian, Ubuntu, OpenSuse, Gentoo, and Fedora. On the Windows
platform, RKWard is available in two forms: as a single binary
installer (requires existing installations of
\proglang{R} and \proglang{KDE}) and
as an installation bundle (including \proglang{R} and
essential parts of the \proglang{KDE} SC). At the time of
this writing, the developers lack the resources to support a MacOS X
port, and especially to provide binaries for MacOS X. However, RKWard
has been shown to be compilable and installable on the Mac, and appears
to be mostly functional.

RKWard cannot be loaded from within an \proglang{R}
session, but rather it is started as a stand-alone application with an
embedded \proglang{R} engine. To facilitate the first
steps for new users, a dialog offers the choice to load an existing
workspace, to start with an empty workspace, or to create a new
\code{data.frame} and open that for editing. Also, an overview help-page is
shown in the document area of the main window. Both start-up features
can be turned off.