\section{Conclusion and outlook}
\label{sec:conclusion_summary}
In this article we have introduced the \pkg{RKWard} GUI to \proglang{R}. \pkg{RKWard} provides features ranging
from easy to use dialogs for common statistical procedures targeted at \proglang{R} novices, to advanced
IDE features targeted at \proglang{R} experts.

\pkg{RKWard} aims to empower users of all knowledge levels to make more efficient use of the 
\proglang{R} programming language, while carefully avoiding to lock in users to a specific
GUI solution. In particular, \pkg{RKWard}
\begin{itemize}
 \item Provides full transparency about the \proglang{R} code that is used to carry out tasks.
 \item Avoids introducing \pkg{RKWard}-specific \proglang{R} functions for central functionality (but uses some for output formatting).
 \item Avoids hard dependencies on third-party \proglang{R} packages.
 \item Uses standard \proglang{R} formats \citep[see][]{RDCT2010c} for data storage, and open standards (HTML, PNG, SVG) for storage of output.
\end{itemize}

Future versions of \pkg{RKWard} will continue to add value for both groups of users. Planned features include
an enhanced interface for debugging \proglang{R} code, support for editing more types of data, and the
ability to connect the \pkg{RKWard} GUI to a remote \proglang{R} engine. Perhaps most importantly, \pkg{RKWard} will
gain many new graphical dialogs for manipulation, analysis, and visualization of data. The ability to
develop these dialogs as plugins allows to develop and distribute GUI dialogs
independently of the \pkg{RKWard} core application, allowing any user to help in enhancing \pkg{RKWard}, without in-depth
programming knowledge.
