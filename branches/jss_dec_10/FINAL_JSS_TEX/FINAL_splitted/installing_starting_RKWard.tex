\section{Installation and platform availability}
\label{sec:installing_starting_RKWard}
Contrary to some other \proglang{R} GUIs, such as \pkg{Rcmdr}, \pkg{RKWard} cannot be installed and started as a
regular \proglang{R} add-on package. Rather, it is started as a stand-alone application which embeds the
\proglang{R} engine, and needs to be installed in a platform dependent way, as detailed below\footnote{
  See \url{http://p.sf.net/rkward/download} for an overview and platform specific download links.
}. Besides the
\pkg{KDE} runtime environment and \proglang{R}, \pkg{RKWard} utilizes a growing number of \pkg{R} add-on packages.
However, these do not have to be installed before hand. Rather \pkg{RKWard} will prompt the user to install
missing packages interactively, on an as-needed basis (see Section~\ref{sec:package_management}).

\subsection{Installation on the GNU/Linux platform}
Historically, \pkg{RKWard} originated on the GNU/Linux platform, and binary packages are available for many
major GNU/Linux distributions, including Debian, Ubuntu, OpenSuse, Gentoo, Fedora, but also for other POSIX\footnote{
\url{http://standards.ieee.org/develop/wg/POSIX.html}
}
compliant systems such as FreeBSD.  The exact size of the installation is system dependent. On Debian x86, the
package is currently around 1.5 MB (Megabyte) compressed, and 5.5 MB uncompressed. However, if the \pkg{KDE}
runtime environment is not yet installed, an installation of \pkg{RKWard} may need several hundred MB of disc
space.

On systems which provide up-to-date packages of \proglang{R} and \pkg{KDE}, compilation from source is
generally unproblematic\footnote{
  See \url{http://p.sf.net/rkward/compilling} for details.
}.

\subsection{Installation on Microsoft Windows}
\pkg{RKWard} will run on Windows XP, Windows Server 2003, Vista, and Windows 7. 32-bit binaries are
provided by the project\footnote{
  See \url{http://sourceforge.net/apps/mediawiki/rkward/?title=RKWard\_on\_Windows\#Installing}
}. Users can
choose between a small installer (1.7 MB), which will add \pkg{RKWard} to pre-existing installations of
\proglang{R} and \pkg{KDE}, and an installation bundle, which provides \pkg{RKWard}, \proglang{R}, and
\pkg{KDE}. This bundle just needs to be unpacked to any user-writable folder, and can be run without any
further steps of installation. When using this bundle, \pkg{RKWard} can also be installed to removable storage
devices (e.\,g. USB sticks) and shared between systems. Its configuration settings are stored in the user's
home directory, and will not be shared across systems, unless the user takes further steps. The size of the
current installation bundle is 132 MB compressed, and around 670 MB installed.

Source installation on the Microsoft Windows platform is comparatively difficult, since various tools need to
be installed\footnote{
  See \url{http://sourceforge.net/apps/mediawiki/rkward/?title=RKWard\_on\_Windows/Packaging} for details.
}.

\subsection{Installation on Mac OS X}
At the time of this writing, the developers lack the resources to support a Mac OS X port, and especially
to provide binaries for Mac OS X. Although \pkg{RKWard} has been successfully compiled and installed on the Mac, and
appeared to be mostly functional, there have also been unresolved reports of failure to compile or start
\pkg{RKWard} on Mac OS X. Since the \pkg{KDE} project currently does not offer binaries for Mac OS X,
installation of \pkg{RKWard} also requires compilation of the \pkg{KDE} runtime environment and its
dependencies from source, which takes many hours to complete on current systems. Further, \pkg{RKWard}'s
graphics device window related features (see Section~\ref{sec:plot_previews}) are only available when
compiling and using \pkg{KDE} and \pkg{RKWard} in \pkg{X11} mode. In conclusion, \pkg{RKWard} on Mac OS X is
not suitable for most users in its current state.

\subsection[Starting RKWard]{Starting \pkg{RKWard}}
\pkg{RKWard} cannot be loaded from within an \proglang{R}
session, but is rather started as a stand-alone application with an
embedded \proglang{R} engine. To facilitate the first
steps for new users, a dialog offers the choice to load an existing
workspace, to start with an empty workspace, or to create a new
\code{data.frame} and open that for editing. Also, an overview help page is
shown in the document area of the main window. Both these start-up features
can be turned off.
