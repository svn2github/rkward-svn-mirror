\section{Installation and platform availability}
\label{sec:installing_starting_RKWard}
Contrary to some other \proglang{R} GUIs, such as \pkg{Rcmdr}, \pkg{RKWard} cannot be installed and started as a
regular \proglang{R} add-on package. Rather, it is started as a stand-alone application which embeds the
\proglang{R} engine, and needs to be installed in a platform dependent way, as detailed, below\footnote{
  See http://p.sf.net/rkward/download for an overview and platform specific download links.
}. Besides the
\pkg{KDE} runtime environment and \proglang{R}, \pkg{RKWard} utilizes a growing number of \pkg{R} add-on packages.
However, these do not have to be installed before hand. Rather \pkg{RKWard} will prompt the user to install
missing packages, interactively, on an as-needed basis (see Section~\ref{sec:package_management}).

\subsection{Installation on the GNU/Linux platform}
Historically, \pkg{RKWard} originates on the GNU/Linux platform, and binary packages are available for many major
distributions, including Debian, Ubuntu, OpenSuse, Gentoo, Fedora, and also FreeBSD. On systems which
provide up-to-date packages of \proglang{R} and \pkg{KDE}, compilation from source is generally unproblematic\footnote{
  See http://p.sf.net/rkward/compilling for details.
}. The exact size of the installation is system dependent. On Debian x86, the package is currently around 1.5 MB (Megabyte) compressed,
and 5.5MB uncompressed. However, if the \pkg{KDE} runtime environment is not yet installed, an installation of \pkg{RKWard} may
need several hundred MB of disc space.

\subsection{Installation on Microsoft Windows}
\pkg{RKWard} can be used on Windows XP, Windows Server 2003, Vista, and Windows 7. Source installation on the
Microsoft Windows platform is comparatively difficult, since various tools need to be installed
\footnote{
  See http://sourceforge.net/apps/mediawiki/rkward/index.php?title=RKWard\_on\_Windows/Packaging for details.
}
, however, 32bit binaries are also provided by the project. The user has a choice between an small installer (1.7 MB),
which can be used to install \pkg{RKWard} to pre-existing installations of \proglang{R} and \pkg{KDE}, and a installation
bundle, which includes \pkg{RKWard}, \proglang{R}, and \pkg{KDE}. This bundle can be unpacked to any user-writable folder,
and can be run without any further steps of installation. Using this method of installation,
\pkg{RKWard} can also be installed to a removable storage medium, and moved between different systems (configuration
settings are stored in the user's home directory, and will not be shared across systems, unless the user takes further steps).
The size of the current installation bundle is 132 MB compressed, and around 670 MB installed.

\subsection{Installation on Mac OS X}
At the time of this writing, the developers lack the resources to support a Mac OS X port, and especially
to provide binaries for Mac OS X. \pkg{RKWard} has been successfully compiled and installed on the Mac, and
appeared to be mostly functional, but there have also been unresolved reports of failure to compile or to start
\pkg{RKWard} on Mac OS X. Since \pkg{KDE} project does not currently offer binaries for Mac OS X, installation
of \pkg{RKWard} also requires compilation of the \pkg{KDE} runtime environment, and its dependencies from source,
which takes many hours to complete on current systems. Further, \pkg{RKWard}'s grapics device window related features
(see Section~\ref{sec:plot_previews}) are only available when compiling and using \pkg{KDE} and \pkg{RKWard} in
\pkg{X11} mode. In conclusion, \pkg{RKWard} on Mac OS X cannot currently be considered ready for regular users.

\subsection[Starting RKWard]{Starting \pkg{RKWard}}
\pkg{RKWard} cannot be loaded from within an \proglang{R}
session, but rather it is started as a stand-alone application with an
embedded \proglang{R} engine. To facilitate the first
steps for new users, a dialog offers the choice to load an existing
workspace, to start with an empty workspace, or to create a new
\code{data.frame} and open that for editing. Also, an overview help page is
shown in the document area of the main window. Both start-up features
can be turned off.
