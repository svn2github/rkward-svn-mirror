\section{Conclusion and outlook}
\label{sec:conclusion_summary}
In this article we have introduced the RKWard GUI to \proglang{R}. RKWard provides features ranging
from easy to use dialogs for common statistical procedures, targetted at \proglang{R} novices, to advanced
IDE features targetted at \proglang{R} experts.

RKWard tries to empower users of all knowledge levels to make more efficient use of the 
\proglang{R} programming language, while carefully avoiding to lock in users to a specific
GUI solution. In particular, RKWard
\begin{itemize}
 \item provides full transparency about the \proglang{R} code that is used to carry out tasks.
 \item avoids introducing RKWard-specific \proglang{R} functions for central functionality (but uses some for output formatting).
 \item avoids hard dependencies on third-party \proglang{R} packages.
 \item uses standard \proglang{R} formats \citep[cf.][]{RDCT2010c} for data storage, and open standards (\proglang{HTML}, \proglang{PNG}, \proglang{SVG}) for storage of output.
\end{itemize}

%% TF: I don't think this comparison is entirely fair. Keep in mind that this is is a special issue about R GUIs.
%% So all those GUIs will base their calculations on R. But some will do it more transparently than others.
%The RKWard development 
%does not focus on \proglang{R} package development, except those internally 
%required for RKWard, but keeps it at the \proglang{R} community. This design brings the intrinsic 
%benefit of highly accurate results since calculations entirely rely on \proglang{R} code. 
%Comparison of the commonly used spreadsheet applications 
%regarding estimation, random number generation and statistical distributions revealed serious 
%limitations. \proglang{R} in contrast was found to be a reliable and accurate statistical 
%software package \citep{Almiron2009, Almiron2010}.
%------------------------------------------------------------------------------------------------
%SR: okay I didn't make my point here. The reference to Almiron2009 and Almiron2010 has just the
%purpose to support that R is the powerhorse and using RKWard does empower the user ultimately.
%Maybe it should be put somewehere else. The question is ``Why should I use RKWard'' and not, let's say MS Office,
%during my research work. There is scientific proof according to the papers why using R and thus RKWard is wise.
%Certainly this applies to all R GUIs, but we are the one with the familiar spreadsheet-like interface and so on.
%Methinks in the background section would be a good place. I seriously think that *some* readers are not aware of this issue and its implications.
%---------------------
%TF: Ok, so it would need to be added to the background section, since that is the only place, where we make
%statements about R itself. I'm reluctant to add too much more to that introductory paragraph, though.
%Perhaps something like 'has been shown to be more reliable and accurate than many competing statistical
%software solutions \citep{Almiron2009, Almiron2010}'. If at all.

Future versions of RKWard will continue to add value for both groups of users. Planned features include
an enhanced interface for debugging \proglang{R} code, support for editing more types of data, and the
ability to connect the RKWard GUI to a remote \proglang{R} engine. Perhaps most importantly, RKWard will
gain many new User Interface dialogs for manipulation, analysis, and visualization of data. The ability to
develop these dialogs as plugins allows to develop and distribute GUI dialogs
independently of the RKWard core application, allowing any user to help in enhancing RKWard, without in-depth
programming knowledge.

\section{Acknowledgments}
\label{sec:acknowledgments}
The software RKWard, described in this paper, is currently developped by Thomas Friedrichsmeier (lead developer), Prasenjit Kapat, Meik Michalke,
and Stefan R\"odiger. Many more people have contributed, or are still contributing to the project in various forms. We would like to
thank (in alphabetical order) Adrien d'Hardemare, Daniele Medri, David Sibai, Detlef Steuer, Germ\'an M\'arquez Mej\'ia,
Ilias Soumpasis, Jannis Vajen, Marco Martin, Philippe Grosjean, Pierre Ecochard, Ralf Tautenhahn, Roland Vollgraf, Roy Qu,
Yves Jacolin, and many more people on \url{rkward-devel@lists.sourceforge.net} for their contributions.
