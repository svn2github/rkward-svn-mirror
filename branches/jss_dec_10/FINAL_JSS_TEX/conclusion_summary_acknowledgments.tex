\section{Conclusion and outlook}
\label{sec:conclusion_summary}
In this article we have introduced the RKWard GUI to \proglang{R}. RKWard provides features ranging
from easy to use dialogs for common statistical procedures, targetted at \proglang{R} novices, to advanced
IDE features targetted at \proglang{R} experts.

RKWard tries to empower users of all knowledge levels to make more efficient use of the 
\proglang{R} programming language, while carefully avoiding to lock in users to a specific
GUI solution. In particular, RKWard
\begin{itemize}
 \item provides full transparency about the \proglang{R} code that is used to carry out tasks.
 \item avoids introducing RKWard-specific \proglang{R} functions for central functionality (but uses some for output formatting).
 \item avoids hard dependencies on third-party \proglang{R} packages.
 \item uses standard \proglang{R} formats \citep[cf.][]{RDCT2010c} for data storage, and open standards (\proglang{HTML}, \proglang{PNG}, \proglang{SVG}) for storage of output.
\end{itemize}

%% TF: I don't think this comparison is entirely fair. Keep in mind that this is is a special issue about R GUIs.
%% So all those GUIs will base their calculations on R. But some will do it more transparently than others.
%The RKWard development 
%does not focus on \proglang{R} package development, except those internally 
%required for RKWard, but keeps it at the \proglang{R} community. This design brings the intrinsic 
%benefit of highly accurate results since calculations entirely rely on \proglang{R} code. 
%Comparison of the commonly used spreadsheet applications 
%regarding estimation, random number generation and statistical distributions revealed serious 
%limitations. \proglang{R} in contrast was found to be a reliable and accurate statistical 
%software package \citep{Almiron2009, Almiron2010}.

Future versions of RKWard will continue to add value for both groups of users. Planned features include
an enhanced interface for debugging \proglang{R} code, support for editing more types of data, and the
ability to connect the RKWard GUI to a remote \proglang{R} engine. Perhaps most importantly, RKWard will
gain many new UI dialogs for manipulation, analysis, and visualization of data. The ability to
develop these dialogs as plugins allows to develop and distribute GUI dialogs
independently of the RKWard core application, allowing any user to help in enhancing RKWard, without in-depth
programming knowledge.

\section{Acknowledgments}
\label{sec:acknowledgments}
The work described in this paper was supported by YOUR NAME OR THE NAME
OF SOMEBODELSE HERE
