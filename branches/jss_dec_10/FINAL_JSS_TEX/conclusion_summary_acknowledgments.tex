\section{Conclusion and outlook}
\label{sec:conclusion_summary}
In this article we have introduced the RKWard GUI to \proglang{R}. RKWard provides features ranging
from easy to use dialogs for common statistical procedures, targetted at \proglang{R} novices, to advanced
IDE features targetted at \proglang{R} experts.
The RKWard development 
does not focus on \proglang{R} package development, except those internally 
required for RKWard, but keeps it at the \proglang{R} community. This design brings the intrinsic 
benefit of highly accurate results since calculations entirely rely on \proglang{R} code. 
Comparison of the commonly used spreadsheet applications 
regarding estimation, random number generation and statistical distributions revealed serious 
limitations. \proglang{R} in contrast was found to be a reliable and accurate statistical 
software package \citep{Almiron2009, Almiron2010}.
\proglang{R} scripts which are routinely in use or require complex steps can be programmed as GUI 
elements in RKWard. Due to its plugin architecture RKWard provides the basis 
for a highly customizable \proglang{R}-GUI. We show that the plugin development can take place 
independently without changes in the core application or compilation of binaries. 
Current developments also targets plugin development via external canals like 
GHNS (Get Hot New Stuff)\footnote{\url{http://ghns.freedesktop.org/}} independently of the 
RKWard plugin development. 
Basically all \proglang{R} functions can be provided with GUIs.
The exclusive use of standard-complient formats (e.g. \proglang{HTML}, PNG) 
and sole use of \proglang{R} formats (\proglang{R} Data Files \citep{RDCT2010c}) 
guarantees cross-platform interoperability.
Future versions of RKWard will continue to add value for both groups of users. Planned features include
an enhanced interface for debugging \proglang{R} code, support for editing more types of data, and the
ability to connect the RKWard GUI to a remote \proglang{R} engine. Perhaps most importantly, RKWard will
gain many new UI dialogs for manipulation, analysis, and visualization of data. The ability to
develop these dialogs as plugins allows any contributor or user to help in enhancing RKWard, without in-depth
programming knowledge.

\section{Acknowledgments}
\label{sec:acknowledgments}
The work described in this paper was supported by YOUR NAME OR THE NAME
OF SOMEBODELSE HERE